
\documentclass[11pt]{article}
\usepackage{digra}
 
\usepackage[authordate,minnames=1,maxnames=2,maxbibnames=10,minbibnames=7]{biblatex-chicago}
\AtEveryBibitem{\clearfield{doi}\clearfield{urlyear}\clearfield{urlmonth}\clearfield{urlday}}
\DeclareFieldFormat[game]{title}{\mkbibemph{#1}}
\DefineBibliographyStrings{english}{%
  references = {},
}

%\usepackage[colorlinks=false]{hyperref}
\usepackage[bookmarksnumbered,pdfpagelabels=true,plainpages=false,colorlinks=true,linkcolor=black,citecolor=black,urlcolor=blue,hyperfootnotes=false,nesting=true]{hyperref}

\newcommand{\hwlcomment}[1]{$\spadesuit$\textsc{#1}$\spadesuit$}
\renewcommand{\hwlcomment}[1]{}

\addbibresource{bibliography.bib}

\title{\addvspace{-2\baselineskip}The Jomini Engine: a historical MMORPG framework}
\author{Hans-Wolfgang Loidl}
\affil{
% School of Mathematical and Computer Sciences, 
Heriot-Watt University, 
Edinburgh EH144AS \\
% telephone \\
\texttt{H.W.Loidl@hw.ac.uk} }

\author{Sandy Louchart}
\affil{
% Digital Design Studio, The Glasgow School of Art, The Hub, Pacific Quay, G51 1EA, UK
%Digital Design Studio,
Glasgow School of Arts, 
Glasgow G511EA \\
%Address line 1 \\
%Address line 2 \\
%telephone \\
\texttt{S.Louchart@gsa.ac.uk}}
\date{\vspace{-60pt}}

\begin{document}
\pagenumbering{gobble} 
\newpage
\pagenumbering{arabic}  
\addvspace{-1\baselineskip}
   \maketitle
   \addvspace{-1\baselineskip}
    \copyrightnotice
    
   \section*{ABSTRACT}

   This short paper discusses the design of the \href{http://www.macs.hw.ac.uk/~hwloidl/Projects/JominiEngine/}{JominiEngine}, a serious game engine for
   massive multi-player online role-playing games (MMORPG), designed to be an educational
   tool for the learning domain of history.
   Main design principles of the game engine are
   \emph{accuracy} in the historic model,
   \emph{flexibility} in the scope of the content modeling, to cover a wide range of historic periods,
   \emph{cooperative team-play embedded in a competitive game}, reflecting the  historical context, and
   high \emph{security} in the interaction with the underlying game engine.
   
%% Place your abstract here. Your abstract goes here. In this paper we describe the formatting
%% requirements for DiGRA Conference Proceedings, and offer recommendations on writing
%% for the worldwide DiGRA readership.
\section*{Keywords}
Serious Games, Game Design, Historical Games, MMORPG

\subsection*{INTRODUCTION}

% motivation of a Serious Game in History
The learning domain of history is attractive to a wide range of audiences, from
young school children to senior citizens, finds particularly strong resonance in the
U.K. where high quality TV productions often attract significant interest and shape a
well informed audience. The means for transferring knowledge, however, hasn't changed
much over the past decades, and most notably is uni-directional and very limited in
the interaction that is possible for the customer. For these reasons, we believe that
a new approach to history teaching is called for that focuses on the notion of
\emph{``interactive history''}: providing an interactive framework in which the
player can interact with a precisely modeled world, and explore the effects
this interaction has on the world and its players.

% core design principles in the game design
We have designed a game engine for  MMORPGs that can serve as a platform for such interactive history.
Its first instantiation  lets the player assume the role of a noble in Britain
between 1194--1214, manage his/her estate, interact with other players, and perform sieges and battles.
%
In our game and system design we follow these overall design principles. \emph{Accuracy\/}
of the modeled world is crucial in order to provide a convincing environment that reflects
core concepts of society in a certain time period. \emph{Flexibility} is generally important
for a  game \emph{engine}, that should be able to model a range of different time periods
allowing for very different game mechanics. Core to the learning experience is the embedding 
of \emph{co-operative} team-play into a game engine that is largely based on a historical, \emph{competitive} model,
in order to learn about social interactions and dynamics, but also about limitations to communication,
in the particular time period. And finally on the system side, we emphasise a high level
of \emph{security} that needs to be built into the core game engine, securing the communication 
between server and clients, and preventing players from corrupting the game data.

% other issues to mention:
% the importance of a \emph{moderated} game, where an experience player acts as a ``herald''
% advising other players about proper in-game conduct
% usefulness of open source, built-from-scratch game engine, over modding of an existing engin
% importance of easy content authoring.

% % status of our work so far; maybe move this to conclusion
% This work is part of a longer-term initiative to build a complete game engine, matching these
% requirements, and deploying it as a learning tool to interested partners. At this stage, we
% have a working prototype of a core game engine, that models mainland Britain as 207 fiefs on a hex-map 
% and is populated
% with a realistic database of 1889 PCs and NPCs. The current implementation covers
% core game concepts of finance management, household management and conflict management, and
% is set in the time period from 1194--1214.  The choice of this time period was mainly motivated
% by the availability of the NPC database, that has been generated by a co-author in the context
% of a different project. The technical details of the game engine are discussed in more detail
% in~(\cite{GALA15}) and the web page for this initiative  provides more background information~(\cite{JominiEngineURL}).

\hwlcomment{fix citations}

\hwlcomment{more on the type of game early: players as nobles that manage fiefs for income, raise armies and
  wage war? But: the engine itself is more general than this specific instance.}

%% Place your text here. Your text goes here. This format is to be used for submissions that
%% are published in the electronic conference proceedings for DiGRA conferences\endnote{ The format was developed for DiGRA 2011, modified slightly for DiGRA Nordic 2012 and FDG/DiGRA 2016, and may change for future conferences. A slightly different format may apply for the forthcoming DiGRA journal.}
%% . The
%% same format will be used for conference articles uploaded to the DiGRA library.

%% In essence, you should format your paper exactly like this document. The easiest way to
%% do this is simply to download this template from the conference web site, and replace the
%% content with your own material.

\section*{BACKGROUND}

\hwlcomment{Cover core related work here; some taken from the GALA paper; should also
  cover key bg on Game Design for this paper}

% should cover: game design, serious games for education in history; conflict simulation


% From GALA15 paper: this focuses on historic wargames; the scope should be broader here: historic simulations
%% Historic Wargames (HW) represent a significant part of
%% well-established game traditions that includes both wargames and war
%% simulations. The wargame genre pre-dates digital gaming and has
%% traditionally been rooted in analogue game-play, such as the
%% historical simulations developed by Avalon Hill [18] and the
%% fantasy/sci-fi simulationsbattle worlds of Warhammer [19]. The recent
%% digitization of the genre has opened up new avenues, possibilities and
%% a renewed appeal to current generations of gamers. Beside its
%% entertainment values, Kirschenbaum [20,17] points to wargames as
%% relevant tools for teaching and reinforcing creative decision-making;
%% a core aspect of wargaming, recognised since Georg von Reisswitz first
%% introduced his wargaming rules in the early 19th Century [21,18].  HWs
%% represent a popular genre and an increasing number of commercial
%% history-based games have become available, many based on historical
%% conflicts.  These games often contain a high degree of historical
%% accuracy and vary both in, firstly, their level of abstraction (from
%% ‘traditional’ wargames in which units are abstractly represented as
%% counters on a hexagon map, to those that use cutting edge animation
%% technology to allow the player to participate in battles) and,
%% secondly, scale (from the modeling of a single battle, to theto the
%% simulation of entire civilisations and time periods).  HW’s also often
%% include role-playing elements; for example, the Crusader Kings series
%% (which uses Paradox’s ClausewitzEngine) [22,19]. Although, not
%% historically accurate, Clash of Clans [23,20], a ‘freemiuim’ mobile
%% strategy MMO, was ranked 3rd (in 2013) in the global list of
%% iOS/Android apps revenues [21].

% The potential of serious games as a learning tool in the domain
% of history has been acknowledged by many oauthors.
Our characterisation of historical games can be summarised
in Kirschenbaum's~(\cite*{Kirschenbaum}) words (on historic wargames) as ``a vehicle for its participants,
either through role-playing or the arbitrary rule-based constraints of
the game world, to critically examine their own assumptions and
decision-making processes''. 
This covers historical games, as well as classical wargames and increasingly
also a broader class of simulations in the domains of management and cultural heritage. 
Anderson~(\cite*{Anderson}) investigated the use of serious games
in the field of cultural heritage and drew attention to the increasing
provision of modding tools as an intrinsic part of many commercial
games, allowing for them to be adapted for educational purposes. 
In our case we emphasise accuracy and flexibility of the game, and therefore
prefer building a game engine from scratch to be able to modify all components.
% From
% a design perspective, accuracy is a key feature for any wargame or
% historical simulation.

The design of our game engine and its historical accuracy has been
strongly influenced by the design of the \textit{The Hundred Years War}
by Dunnigan and Nofi~(\cite*{hyw}). This game  provides an entire on-line textbook by an
academic as historical background for the game. Furthermore,
Dunnigan~(\cite*{Dunnigan})  argues for realism in historical wargames but
also makes a case for balancing historical accuracy with fun gameplay
and promotes the notion of ``dynamic potential'',  i.e.\ mechanisms
through which the player might interact with the game world in order
to alter history, in order to better integrate both aspects of the
design. 
Sabin~(\cite*{Sabin}) argues even more strongly for simplicity in conflict simulations,
by using simple game mechanics to its fullest  in what he calls micro-games.

% Whilst this places an extra burden on the game designer who
% needs to ensure accuracy, both in game data and mechanisms, it allows
% for a contextual positioning of relevant aspects such as geography,
% society, and military technology and doctrine.

Several modern, commercial games demonstrate the feasibility of building an accurate game model:
%  provide remarkable detail in the game model, : 
Paradox's~(\cite*{ck2}) ClausewitzEngine in \textit{Crusader Kings\/} provides
an impressive NPC database with detailed family connections, and
emphasises  role-playing elements that reflect historical reality,
such as the importance of family management and its interaction with the
larger political process.
%% ; for example, the  series
%% (which uses Paradox’s ClausewitzEngine) [22,19]

%% The DiGRA proceedings are formatted for a US Letter format, single column page. The
%% size is chosen to allow for on-screen readability. On each page your material (not
%% including the page number) should fit within a rectangle of 14 x 22 cm (5.5 x 8.75 in.),
%% centered on a US letter page, beginning 2.54 cm (1 in.) from the top of the page. On an
%% A4 page, use a text area of the same dimensions, again centered. Right margins should be
%% justified, not ragged. Beware that Word can change these dimensions in unexpected
%% ways.

\section*{GAME DESIGN}

% that can serve as an educational tool in history,

The aforementioned game design principles are motivated by our long-term vision of
a game engine for ``interactive history'', capturing diverse
aspects such as the political system, the economy, the societal context, and
conflict resolution, essentially wargaming.
By abstracting over the concrete game mechanics, we gain flexibility over all 
of these aspects, and can study them in a historical or an idealised context.
%
% not only in
% terms of time-period, but also in terms of focus, so that concrete instances of
% the game engine can be used to study specifc aspects, e.g.\ the societal model.

An inherent tension in such a design is between accuracy, or realism, and playability, or engagement.
We have opted to focus on accuracy in order to build solid foundations for the delivery
of insights about the modeled world. However, we do not expect all of the underlying detail
to be tunable, or even directly visible, to the player. We therefore focus on simple,
high-level game mechanics, and translate these into a more complex system of configurable rules 
in the game engine to drive the simulation. 
It also allows us as game designers to re-visit the low-level rules, in an effort to
improve the model, without necessarily disrupting the high-level game mechanics, and
in this way learn ourselves about characteristics of the domain. As McCarthy~(\cite*{McCarthy}) states about wargame simulations,
``Unfix [the terms of engagement] $\ldots$ and the simulation becomes a modeling exercise.
Thus simulation crosses over into modeling when the constants of the system become variables.''
This highlights an important long-term issue for us. The game engine can be used not only as a vehicle of interactive teaching, 
but also as the object of study itself, giving insights on the modeling of history
and also, from a systems point of view, teaching lessons in the design of large-scale software.

We chose a \emph{role-playing game\/}  to put players into an immersive historical setting and to encourage player interaction.
% as the most natural class of games that supports
% the player interaction to be expected in a historical setting. 
In particular, for any larger scale activity, such as a siege, players would need to interact
and form teams, with the team leader, often the King, having to ensure loyalty.
To this end, we classified the available resources along several dimensions,
such as (non-)transferable, (non-)replenishable etc, and we provide a hierarchical
model of PCs in the game, typically with the King at the root of the hierarchy.
Game resources are allocated asymmetrically to PCs on different levels of the
hierarchy. To compensate for the ``unfairness'' in this allocation, a prestige-system
is used in the game to track a player's achievements, separately from tracking
the accumulation of tradeable resources.

In the current design the JominiEngine takes a \emph{strategic view\/} (or macro-history view) of the world,
modeling mainland Britain as a hex-map of 207 fiefs, and providing an NPC
database with 1889 entries. In this view, resource management issues and
conflict resolution take primary roles, as expected from a historical simulation.
However, we also put strong emphasis on the household and family management
component, to give the player a more tangible concept at the centre of the planning,
and to highlight the issue of dynastic planning over a longer time frame.
%
% probably less important, but would be a nice balance to the classic view
%
This strategic view should in the long-term be complemented with a more
tactical view that should allow the player to ``zoom in'', for example
into a particular fief, in order to manage the planning of its development in
more detail, or into a battle between two armies, to influence its outcome
not only based on the skill levels of the NPCs, but also taking in decisions
about actions in several phases of the battle. 
% A prototype for the latter has
% been developed in a related project, but not yet been integrated into the
% JominiEngine. Such two views help increase player engagement by letting them
% choose their preferred mode of play, and is used in popular commercial
% game engines such as~\textit{Total War}.

% Crucial to the success of such a game engine as educational tool will be the
% balance between learning objectives in the domain of history, and player
% engagement facilitated by historical narratives, a diverse range of player
% actions, and the graphical interface provided to the player. On the latter
% side our current implementation provides a fairly basic interface. However,
% by building the game client on the Unity framework, it has a lot of
% potential in making it more attractive.

Crucial to the success of such a game engine as educational tool will be the
balance between learning objectives in the domain of history, and player
engagement facilitated by historical narratives, a diverse range of player
actions, and the graphical interface provided to the player.
The core narrative is, at the moment, fairly loose and focused on empire and domain
building, taking a macro-history view, and including a tunable notion of prestige.
We however plan to complement this with aspects of micro-history, to allow the player
to see the wider societal context of the world.
On the client
side our current implementation provides a fairly basic interface. However,
by building the game client on the Unity framework, it has a lot of
potential in making it more attractive.


\hwlcomment{need to add more on game narrative and immersive player experience here}


\section*{CONCLUSION}

We have discussed the long-term vision and the main game design aspects in
developing a game engine as an educational tool for the learning domain of history: the \href{http://www.macs.hw.ac.uk/~hwloidl/Projects/JominiEngine/}{JominiEngine}. % cite{JominiEngineURL}.
At this stage, we have a working prototype of a core game engine, as well as an instantiation of the game
that models mainland Britain as 207 fiefs on a hex-map and is populated
with a realistic database of 1889 PCs and NPCs. The current implementation covers
core game concepts of finance management, household management and conflict management, and
is set in the time period from 1194--1214.
The technical details of the game engine are discussed in more detail
in~(\cite{GALA15}) and the \href{http://www.macs.hw.ac.uk/~hwloidl/Projects/JominiEngine/}{web page} for this initiative provides more background information. % ~(\cite{JominiEngineURL})

We view such a flexible game engine, that can model different time periods and can focus on different
aspects of the game world, as a useful complementary tool in history education, making the
learning process more interactive and driven by the learner. By choosing a specific time period
and a  topic, such as noble interaction in a feudal society, the teacher
can describe the general framework for the learning experience, and select from
a range of available narratives to guide the student towards the intended learning outcome.

\hwlcomment{Comment on current size of server, size of DB, and security enhancements}

% While the current version of the JominiEngine is not yet at a deployable stage, the core
% system functionality is available, and secure. 
A focus for future development is an extension of the rule-based low-level game model,
content authoring of a precise historical game world, adding core concepts such as religion, and to
improve the interface of the client to make it more attractive for players.
%
By designing a flexible, OpenSource game engine from scratch we also want to provide a
\emph{``motherboard''} for further studies in using specific game mechanics in this engine,
and we invite interested researchers to co-operate in this process~(\cite{JominiEngineURL}).

\hwlcomment{Comment on macro-vs-micro history; the importance of a \emph{moderated} game, where an experience player acts as a ``herald'' advising other players about proper in-game conduct; usefulness of open source, built-from-scratch game engine, over modding of an existing engine; importance of easy content authoring.}

%% It is important that you write for the DiGRA audience. Please read previous years’
%% Proceedings (available from the DiGRA library, http://www.digra.org/dl) to understand
%% the writing style and conventions that successful authors have used. It is particularly
%% important that you state clearly what you have done, not merely what you plan to do, and
%% explain how your work is different from previously published work, i.e., what is the
%% unique contribution that your work makes to the field? Please consider what the reader 
%% will learn from your submission, and how they will find your work useful. If you write
%% with these questions in mind, your work is more likely to be successful, both in being
%% accepted into the conference, and in influencing the work of our field.

% \section*{ACKNOWLEDGMENTS}

% We would like to thank the Scottish Informatics and Computer Science Alliance (SICSA) for
% their support of the JominiEngine initiative.

% \section*{ENDNOTES}
% \addvspace{-1\baselineskip} %%I can't find another way to reduce the size of the gap for the end notes
% \theendnotes

\section*{BIBLIOGRAPHY}
\vspace{-1em}
{
%\setlength{\parskip}{1pt} 
\begin{small}
\printbibliography
\end{small}
}
\end{document}
